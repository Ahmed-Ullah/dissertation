We are considering the Poisson-Boltzmann Equation (PBE) as the governing equation for a solute macro molecule immersed in an aqueous solvent environment illustrated in Fig 1. Our computational domain $\Omega \in \mathbb{R}^3$ is separated into two regions $\Omega^-$ and $\Omega^+$ by the molecular surface $\Gamma$, which is an arbitrarily shaped dielectric interface. Here $\Omega^-$ is the molecule domain with dielectric constant $\epsilon^-$ and $\Omega^+$ is the solvent domain with dielectric constant $\epsilon^+$. The cubic shape boundary of $\Omega= \Omega^-\cup \Omega^+ $ is denoted by $\delta \Omega$. The charges inside $\Omega^-$ has been distributed as the partial charges to assign to the nearest grid points to the center of each atom inside the molecule. The charges outside in $\Omega^+$ are mobile ions which are described by the Boltzmann distribution. Then the electrostatic interaction of this solute-solvent system for $\textbf{r} \in \mathbb{R}^3$ is governed by the nonlinear Poisson-Boltzmann Equation (PBE) as, 
\begin{equation}
			-\nabla.(\epsilon(\textbf{r})\nabla \phi(\textbf{r}))+\bar\kappa^2(\textbf{r}) \sinh (\phi(\textbf{r}))=\rho(\textbf{r})\label{pbe} %\\
					%\left[\phi \right]_\Gamma = 0 \textnormal{ and } \left[\epsilon\phi_n\right]_\Gamma = 0 
\end{equation}
with the boundary condition,
\begin{equation}
	\phi_b (\textbf{r}) = \frac{e_c^2}{k_B T} \sum_{i=1}^{N_c} \frac{q_i e^{-|\textbf{r}-\textbf{r}_i | \sqrt{\frac{\bar\kappa^2}{\epsilon^+}} }}{\epsilon^{+}|\textbf{r}-\textbf{r}_i|} \label{bd_cond}
\end{equation}
where the singular source $\rho(\textbf{r})$ term is defined as,
\begin{equation}
	\rho(\textbf{r})= 4\pi \frac{e_c^2}{k_B T}\sum_{i=1}^{N_c} q_i \delta(\textbf{r}-\textbf{r}_i) \label{rho}
\end{equation}
There are two conditions on $\Gamma$ needed to be satisfied from the dielectric theory for the potential $\phi$ and flux density $\epsilon \phi_\textbf{n} $, 
\begin{equation}
\left[\phi \right]_\Gamma = 0 \textnormal{ and } \left[\epsilon\phi_n\right]_\Gamma = 0 \label{ju_cond}
\end{equation}
Here $\textbf{n}=(n_x,n_y,n_z)$ is the outer normal direction on the interface $\Gamma$ and $\phi_\textbf{n}= \frac{\partial \phi}{\partial\textbf{n}} $ is the directional derivative in \textbf{n}. The notation $[f]_\Gamma = f^+-f^-$ represent the difference of the functional value across the interface $\Gamma$. 

The dielectric constant $\epsilon$ is piecewise such that, $\epsilon(\textbf{r})=\epsilon^-$ for $\textbf{r} \in \Omega^-$ and $\epsilon(\textbf{r})=\epsilon^+$ for $\textbf{r} \in \Omega^+$. Here $N_c$ is the total number of atoms in the solute molecule, $k_B$ is the Boltzmann constant, $e_c$ is the fundamental charge and $q_i$, in the same unit as $e_c$ is the partial charge on the \textit{i}th atom of the solute molecule located at position $\textbf{r}_j$. The Debey-Huckel parameter $\bar\kappa^2 =\Big(\frac{2N_A e_c^2}{100 k_b T}\Big)I_s =  8.486902807$\AA$^{-2} I_s$ \cite{Holst:1993} for $\textbf{r} \in \Omega^-$ and $\bar\kappa=0$ for $\textbf{r} \in \Omega^+$. Here $N_A$ is Avogadro’s Number and $I_s$ is the molar ionic strength. The reader can refer to REF1 and REF2 for more details about definitions and units of these coefficients. 
\section{Two-component regularization for singular sources}
\label{2_comp_reg}
To avoid the difficulty due to the source term and the work for solving the Laplace equation we consider a two component regularization proposed by Cai, Wang and Zhao \cite{Cai2009}. For this regularization the electrostatic potential $\phi$ has been considered as the addition of the coulmb component $\phi_c$ and the reaction field component $\phi_{RF}$ as $\phi= \phi_c +\phi_{RF}$. Here $\phi_c$ satisfies the following Poission's equation, 
\begin{equation}
	\begin{cases}
		-\epsilon^- \Delta\phi_C(r) = \rho(r) \text{   in   }\mathbb{R}^3 \label{rho_eq} \\
		\text{      }\phi_C(r)= 0. \text{   as  } |r| \rightarrow \infty
	\end{cases}
\end{equation}
which gives us the analytical solution as the Green's function $G$ for $\phi_C$ as,
\begin{equation}
	G(r) = \frac{e_c^2}{k_B T } \sum_{i=1}^{N_c} \frac{q_i }{\epsilon^{-}|r-r_i|} \label{Green} %\text{ where } C = \frac{e_c^2}{k_B T }
\end{equation}
	
%	Now the non-linear PBE from (\ref{pbe}) can be rewritten as,
%	
%	  \begin{eqnarray}
%				-\nabla.(\epsilon\nabla (\phi_C(\textbf{r})))-\nabla.(\epsilon\nabla (\phi_{RF}(\textbf{r})))+\bar\kappa^2(\textbf{r}) \sinh (\phi_C(\textbf{r})+\phi_{RF}(\textbf{r}))&=&\rho(\textbf{r}), \text{ in } \Omega \label{pbe_reg} %\\
%						%\left[\phi \right]_\Gamma = 0 \textnormal{ and } \left[\epsilon\phi_n\right]_\Gamma = 0 
%	\end{eqnarray}
%	
%	The reaction field components can be physically interpreted as the electrostatic field generated by the charges induced by changes of dielectric constant of the  solvent around the solute from $\epsilon^-$ to $\epsilon^+$ \cite{Cai2009}. Now substituting (\ref{rho}) into (\ref{pbe_reg}) we have,
%	\begin{eqnarray}
%	-\nabla(\epsilon^+ \phi_{RF}(\textbf{r})) +\bar\kappa^2 \sinh(\phi_C(\textbf{r})+\phi_{RF}(\textbf{r}))&=& 0 \text{ in } \Omega^- \cup \Omega^+	
%	\end{eqnarray}
%	More explicitly $\phi_{RF}$ satisfies the following elliptic interface problem \cite{Chen2007}, 
%\begin{eqnarray}
%		-\nabla.(\epsilon^- \nabla  \phi_{RF}) &=& 0 \text{ in } \Omega^-\\  
%		-\nabla(\epsilon^+ \phi_{RF}) +\bar\kappa^2 \sinh(\phi_C+\phi_{RF})&=& 0 \text{ in } \Omega^+\\
%		\left[\phi_{RF}\right] &=& 0 \text{ on } \Gamma \\
%		\left[\epsilon\frac{\partial \phi_{RF}}{\partial n}\right]&=& (\epsilon^+-\epsilon^- ) \frac{\partial G}{\partial n} \text{ on } \Gamma\\
%		\phi_{RF}&=& \phi_b-G \text{ on } \partial \Omega \label{rf_sys}
%	\end{eqnarray} 
%	
%So now there is no singular term on the right hand side but we still have several numerical difficulties. In \cite{Holst2010} its been reported that $\phi_C$ and $\phi_{RF}$ have different signs and their magnitude is much larger than that of $\phi$. Here  a relatively small error in $\phi_{RF}$ will produce a relatively larger error in $\phi$ given that the $phi_C$ is analytically calculated. Sometimes this amplifying factor can be as large as $(\epsilon^+/\epsilon^- - 1 )$ \cite{Holst2010}. In our case this factor is about $79$ by taking $\epsilon^+=80$ and $\epsilon^-=1$. Another problem is, the calculation of $\phi_C$ is necessary for all N grid points in $\Omega^+$ at a computational cost $O(N^2)$ which is very expensive for large $N$. 
%
Then Luo and his collaborators \cite{Cai2009} proposed to solve for the original solution $\phi$ in $\Omega^+$ instead of the reaction component $\phi_{RF}$. Now to make the required adjustments to fit this regularization approach \cite{Cai2009} with finite difference and finite element method Zhao and Geng \cite{Geng2017a} proposed a new elliptic problem with discontinuous function flux jumps for the two-component regularization. In particular they defined the regularized potential as,  
	
\begin{eqnarray}
	\tilde{ \phi} &=& \begin{cases}
	\phi_{RF} \text{ in } \Omega^-\\
	\phi_C + \phi_{RF} \text{ in } \Omega^+\\
	\end{cases}
\end{eqnarray}
 
The jump conditions for $\tilde\phi $ were derived from $\phi$ and the definition $\phi=\phi_C+\phi_{RF}$ as,

\begin{eqnarray}
\phi^+=\phi^-_{RF}+\phi^-_C,\text{  and  } \epsilon^+ \frac{\partial \phi^+}{\partial n}=\epsilon^- \frac{\partial \phi^-_{RF}}{\partial n}+\epsilon^- \frac{\partial \phi^-_C}{\partial n} \text{ on } \Gamma
\end{eqnarray}

Thus, the regularized PB equation of $\tilde \phi$ with corresponding interface and boundary conditions are given as. 
\begin{eqnarray}
	-\nabla.(\epsilon^- \nabla \tilde{ \phi}) &=& 0 \text{ in } \Omega^-\\ \label{phitilde1}
	-\nabla.(\epsilon^+ \nabla \tilde{ \phi}) +\bar\kappa^2 \sinh(\tilde{ \phi})&=& 0 \text{ in } \Omega^+\\\label{phitilde2}
	\left[\tilde{ \phi}\right] &=& G \text{ on } \Gamma \\ \label{phitilde3}
	\left[\epsilon\frac{\partial \tilde{ \phi}}{\partial n}\right]&=& \epsilon^-  \frac{\partial G}{\partial n} \text{ on } \Gamma\\\label{phitilde4}
	\tilde{\phi} &=& \phi_b \text{ on } \partial \Omega \label{phitilde5}
\end{eqnarray}	
	


Now from (\ref{phitilde1}) and (\ref{phitilde5}) we can summarize that, 

\begin{eqnarray}
	-\nabla . (\epsilon \nabla \tilde{ \phi}) +\bar\kappa^2 \sinh(\tilde{ \phi})&=& 0 \text{ in } \Omega^-\cup\Omega^+\label{RPB}
\end{eqnarray}
	 
where $\tilde \phi$	 satisfies the PB equation without the source term. Then Zhao and Geng \cite{Geng2017a} proposed to numerically solve the regularized PB interface problem (RPB) given in (\ref{phitilde1})-(\ref{phitilde5}) and to finally recover the original solution as $\phi= \tilde\phi $ in $\Omega^+$ and $\phi = \tilde \phi + G $ in $\Omega^-$, Where the Green's function $G$ is analytically given.
\section{Solvation energy from Electrostatic potential}

The solvation energy can be defined as the energy released when the solute in free space is dissolved in solvent and it can be expressed in terms of the electrostatic free energy $\Delta G_{\rm ele}$ and the coulomb energy $E_{\rm cou}$ as 

\begin{eqnarray}
	\Delta G_{\rm ele}&=& E_{\rm cou}+ E_{\rm sol}\label{eq_Gele}\\
	E_{\rm cou} &=& \sum_{i=1}^{N_c}\sum_{j=1}^{N_c}\frac{q_iq_j}{\epsilon^-d_{i,j}}, i\neq j\label{eq_Ecou}
\end{eqnarray}
where $q_i$ and $q_j$ are the charges at the center of the atoms and $d_{i,j}$'s are the distance between the $i$-th and $j$-the atom. 
  
Then Sharp and Honig in \cite{Sharp_Honig}  described the calculation electrostatic free energy by, 
\begin{equation}
\Delta G_{\rm ele}=\int_{\mathbb{R}^3}\left(\phi\rho+\Delta\Pi-\frac{1}{2}\epsilon|{\bf E}|^2\right)d{\bf r}
\end{equation}

where $\phi$ is the electrostatic potential, $\rho$ is the fixed charge density represented as a smeared surface charge or as a collection of point charges, $\Delta\Pi$ is  the excess osmotic pressure of the mobile ion cloud, and $\frac{1}{2}\epsilon|{\bf E}|^2$ is the electrostatic stress.

Now for the simplification of the numerical validations of our proposed schemes in the next chapter we omitted the energy components related to the mobile ion pressure and the electrostatic stress to report the solvation energy $E_{\rm sol}$ as,  

\begin{eqnarray}
	E_{\rm sol}=\Delta G_{\rm ele}-E_{\rm cou}\approx\sum\limits_{i=1}^{N_c}q_i\phi_{RF} ({\bf r}_i)\label{eq_solvation}
\end{eqnarray}

Here the contribution of the mobile ion pressure and the electrostatic stress to the whole calculation is really small and computationally more challenging. The readers can refer to \cite{GENG_WEI2011,Gilson,Sharp_Honig} for more details.				


\section{Pseudo-transient continuation approach for the PBE}

As the pseudo-transient continuation approach, an indirect method discussed in \cite{zhao_pseudo-time-coupled_2011} \cite{Sayyed-Ahmad2004} \cite{shestakov_solution_2002} introduced a pseudo-time derivative to solve the PB equation as,  

\begin{eqnarray}
	 	\frac{\partial u}{\partial t} &=& \nabla.(\epsilon \nabla u)-\bar\kappa^2 \sinh(u) \text{ in } \Omega^- \cup \Omega^+ \label{tdrpb} \\
	 	  \left[u\right]&=& G,\text{ and } \left[\epsilon\frac{\partial u}{\partial n}\right]= \epsilon^-  \frac{\partial G}{\partial n} \text{ on } \Gamma\\ \label{jump}
	 	  u&=&\phi_b \text{ on } \partial \Omega
	\end{eqnarray}

Here the time independent regularized PB equation in (\ref{RPB}) has been converted in to a time dependent regularized PB (TDRPB) in (\ref{tdrpb}). Our goal is to first specify the initial condition, which could be the electrostatic potential solved from a linearized PB \cite{zhao_pseudo-time-coupled_2011} equation or trivially $u = 0$ and then numerically integrate (\ref{tdrpb}) for a sufficiently long period to get the steady state solution as the solution of the original regularized PB (\ref{RPB}). Here the sign on the right hand side of eq. (\ref{tdrpb}) has been considered as the reverse of the eq.(\ref{RPB}) to ensure the numerical stability. 

But, there are difficulties in the numerical integration of the TDRPB equation (\ref{tdrpb}) because of the requirement of long time integration, when explicit time stepping methods are usually not efficient \cite{Sayyed-Ahmad2004},\cite{shestakov_solution_2002},\cite{zhao_pseudo-time-coupled_2011}, \cite{zhao_operator_2014}. Hence we are employing a semi-implicit time splitting method \cite{Sayyed-Ahmad2004},\cite{shestakov_solution_2002} which have been commonly used to solve the TDPB in the literature. 
