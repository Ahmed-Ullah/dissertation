The pseudo-transient methods and regularization methods are popular methods to solve nonlinear PBE. Even though these two types were successful in circumventing different challenges to solve PBE, each one of them was lacking the advantages of the other one. When we tried to combine them we faced a new challenge due to  the new jump conditions being nonzero. This forced us to find a way to apply interface treatments. The MIB method [ ] was a great choice for this type of interface treatments but it would have ruined the tridiagonal structure of the finite difference operator matrix of the 1D equations like (\ref{GFM-ADI}) (\ref{GFM-ADI2}) and (\ref{GFM-ADI3}) in all of our proposed methods. Having tridiagonal structure for these three equations are very important since we have to solve all three of them at each time step. So for small time step size, it would take unreasonably long time for the whole system to reach the equilibrium state to produce the solution.   

So we considered the GFM method [ ] [ ] for $L=1$ which uses a three point stencil keeping the tridiagonal structure for the 1D equations in our proposed methods. But the original GFM method requires the jump conditions to be in axial directions while the regularized PBE has its jump condition in the normal direction. It motivated us to modify the original GFM  method to able to use the normal direction jump condition by considering approximate jump conditions like (\ref{eq:m-gfm1}) and (\ref{eq:m-gfm2}). 

In comparison with the existing pseudo-transient approaches the GFM-ADI proposed in this dissertation is much more stable than ADI in \cite{geng_fully_2013}. This makes GFM-ADI a very practical method, while the ADI method is impractical by requiring too small time step size. In some sense, the GFM-ADI method is even better than LOD methods in \cite{Wilson2016}. Although LOD methods are unconditionally stable, energies are inaccurate for large time steps. GFM-LODCN and GFMLOD are also producing more accurate results than its predecessor the LOD methods in \cite{Wilson2016} leaving an option for the cases when GFM-ADI fails to converge. 

In future we have a plan to take the advantage of the simplicity of these proposed schemes to the other areas of molecular biophysics like the problem of computing the electrostatic field and forces for molecular dynamic simulations as in \cite{GENG_WEI2011}. 
 