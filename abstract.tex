The Poisson Boltzmann equation (PBE) is a well-established implicit solvent continuum model for the electrostatic analysis of solvated biomolecules. The solution for the nonlinear PBE is still a challenge due to its strong singularity by the source terms, dielectrically distinct regions, and exponential nonlinear terms. In this dissertation, a new alternating direction implicit method (ADI) is proposed for solving the nonlinear PBE using a two-component regularization. This scheme inherits all the advantages of the two-component regularization and the time-dependent PBE with ADI method while possess a novel approach to combine them. A modified version of the one-dimensional ghost fluid method (GFM) has been introduced to incorporate the nonzero jump condition into a new ADI method to propose GFM-ADI method. It produced better results in terms of  spatial accuracy and stability compared to the previous ADI method and simpler to implement by circumventing the work necessary to apply the MIB method with the regularization for a 3D problem. Moreover, the stability of the GFM-ADI method has been significantly improved in comparing with the non-regularized ADI method, so that stable protein simulations can be carried out with a pretty large time step size. Two locally one-dimensional (LOD) methods have also been developed for the time-dependent regularized PBE, which are unconditionally stable.  
%To continue the  search for more stable methods this modified GFM method and Locally One Dimensional (LOD) method has been combined similarly with Implicit Euler method and Crank-Nicolson method two propose GFM-LODCN and GFM-ODIE method. 
%{\color{red}Though this scheme can use larger time increments than the previous ADI method, it still blows up for large time increments. Later to address this issue with the stability, Locally One Dimensional (LOD) method has been used to replace the ADI method as the operator splitting part.} 
Finally, for numerical validation, we have evaluated the solvation free energy for a collection of 24 proteins with various sizes and the salt effect on the protein-protein binding energy of the complex 1beb.