To generate the molecular surface as the interface in our computational domain we had to use the 3D structural data for proteins and nucleic acids available in the Protein Data Bank(PDB). PDB contains freely accessible data on the internet submitted by biologists and biochemists from around the world. These data are usually acquired by  X-ray crystallography, NMR spectroscopy, or, cryo-electron microscopy. Each molecule is represented by 4 letter  .Among several types of data file available from the PDB we focused on \textit{.pdb} type files available from the website of RCSB \cite{RCSB}, one of the member organization of the PDB. 
\begin{table}[!ht]
\begin{tabular}{|l|l|l|l|l|}
\hline
Record Type             & Columns & Data                            & Justification & Data Type  \\ \hline
\multirow{15}{*}{ATOM}  & 4-Jan   & “ATOM”                          &               & character  \\ \cline{2-5} 
                        & 7-11  & Atom serial number              & right         & integer    \\ \cline{2-5} 
                        & 13-16   & Atom name                       & left*         & character  \\ \cline{2-5} 
                        & 17      & Alternate location indicator    &               & character  \\ \cline{2-5} 
                        & 18-20§  & Residue name                    & right         & character  \\ \cline{2-5} 
                        & 22      & Chain identifier                &               & character  \\ \cline{2-5} 
                        & 23-26   & Residue sequence number         & right         & integer    \\ \cline{2-5} 
                        & 27      & Code for insertions of residues &               & character  \\ \cline{2-5} 
                        & 31-38   & X orthogonal Å coordinate       & right         & real (8.3) \\ \cline{2-5} 
                        & 39-46   & Y orthogonal Å coordinate       & right         & real (8.3) \\ \cline{2-5} 
                        & 47-54   & Z orthogonal Å coordinate       & right         & real (8.3) \\ \cline{2-5} 
                        & 55-60   & Occupancy                       & right         & real (6.2) \\ \cline{2-5} 
                        & 61-66   & Temperature factor              & right         & real (6.2) \\ \cline{2-5} 
                        & 73-76   & Segment identifier¶             & left          & character  \\ \cline{2-5} 
                        & 77-78   & Element symbol                  & right         & character  \\ \hline
79-80                   & Charge  &                                 & character     &            \\ \hline
\multirow{2}{*}{HETATM} & 6-Jan   & “HETATM”                        &               & character  \\ \cline{2-5} 
                        & Jul-80  & same as ATOM records            &               &            \\ \hline
\multirow{6}{*}{TER}    & 3-Jan   & “TER”                           &               & character  \\ \cline{2-5} 
                        & 7-11  & Serial number                   & right         & integer    \\ \cline{2-5} 
                        & 18-20§  & Residue name                    & right         & character  \\ \cline{2-5} 
                        & 22      & Chain identifier                &               & character  \\ \cline{2-5} 
                        & 23-26   & Residue sequence number         & right         & integer    \\ \cline{2-5} 
                        & 27      & Code for insertions of residues &               & character  \\ \hline
\end{tabular}
\caption{Protein Data file (\textit{.pdb} type) format}
\label{tab:PDB_format}
\end{table}
%TODO have to describe the format of pdb file then apbs type the pqr type. 