\section{Two-component regularization for singular sources}
\label{2_comp_reg}
To avoid the difficulty due to the PBE for the vacuum case Cai, Wang and Zhao \cite{Cai2009} proposed a two component regularization. For this regularization the electrostatic potential $\phi$ is expressed as the sum of the coulomb component $\phi_c$ and the reaction field component $\phi_{RF}$.
% as $\phi= \phi_c +\phi_{RF}$. 
Here $\phi_c$ satisfies the following Poisson's equation, 
\begin{equation}
	\begin{cases}
		-\epsilon^- \Delta\phi_C({\bf r}) = \rho({\bf r}) \text{   in   }\mathbb{R}^3 \label{rho_eq}, \\
		\text{      }\phi_C({\bf r})= 0. \text{   as  } |{\bf r}| \rightarrow \infty.
	\end{cases}
\end{equation}\label{eq:cl_comp}
which has the analytical solution as the Green's function $G$ for $\phi_C$ as,
\begin{equation}
	G({\bf r}) = \frac{e_c^2}{k_B T } \sum_{i=1}^{N_c} \frac{q_i }{\epsilon^{-}|{\bf r}-{\bf r}_i|}. \label{eq:green} %\text{ where } C = \frac{e_c^2}{k_B T }
\end{equation}
Now the PBE from equation (\ref{pbe}) can be rewritten as,
\begin{equation}
		-\nabla.(\epsilon\nabla (\phi_C(\textbf{r})))-\nabla.(\epsilon\nabla (\phi_{RF}(\textbf{r})))+\bar\kappa^2(\textbf{r}) \sinh (\phi_C(\textbf{r})+\phi_{RF}(\textbf{r}))=\rho(\textbf{r}). \label{pbe_reg} %\\
				%\left[\phi \right]_\Gamma = 0 \textnormal{ and } \left[\epsilon\phi_n\right]_\Gamma = 0 
\end{equation}
The reaction field component $\phi_{RF}$ can be physically interpreted as the electrostatic field generated by the charges induced by replacing the solvent around the solute molecule. It changes the dielectric constant in $\Omega^+$ form $\epsilon^+$ to $\epsilon^-$ \cite{Cai2009}. Now substituting (\ref{eq:cl_comp}) into (\ref{pbe_reg}) we have,
\begin{eqnarray}
-\nabla(\epsilon^+ \phi_{RF}(\textbf{r})) +\bar\kappa^2 \sinh(\phi_C(\textbf{r})+\phi_{RF}(\textbf{r}))&=& 0 \text{ in } \Omega^- \cup \Omega^+.	
\end{eqnarray}
More explicitly, $\phi_{RF}$ satisfies the following elliptic interface problem as described in \cite{Chen2007}: 
\begin{eqnarray}
		-\nabla.(\epsilon^- \nabla  \phi_{RF}) &=& 0 \text{ in } \Omega^-,\\  
		-\nabla(\epsilon^+ \phi_{RF}) +\bar\kappa^2 \sinh(\phi_C+\phi_{RF})&=& 0 \text{ in } \Omega^+,\\
		\left[\phi_{RF}\right] &=& 0 \text{ on } \Gamma, \\
		\left[\epsilon\frac{\partial \phi_{RF}}{\partial n}\right]&=& (\epsilon^+-\epsilon^- ) \frac{\partial G}{\partial n} \text{ on } \Gamma,\\
		\phi_{RF}&=& \phi_b-G \text{ on } \partial \Omega ,\label{rf_sys}
\end{eqnarray} 
Now there is no singular term on the right hand side but we still have several numerical difficulties left to address. It is known that $\phi_C$ and $\phi_{RF}$ have different signs and their magnitude is much larger than $\phi$ \cite{Holst2010}. It can be shown \cite{Holst2010} using the benchmark problem called Born Ion that, a relatively small error in $\phi_{RF}$ will produce a relatively larger error in $\phi$ given that the $\phi_C$ is analytically calculated. Sometimes this amplifying factor \cite{Holst2010} can be as large as $(\epsilon^+/\epsilon^- - 1 )$. In our case this factor is about $79$ by taking $\epsilon^+=80$ and $\epsilon^-=1$. Another problem is the calculation of $\phi_C$ is necessary at all N grid points in $\Omega^+$ at a computational cost $O(N^2)$ which is very expensive for large $N$. 

These difficulties motivated us to consider a newer version of the two component regularization proposed by Luo et al. in \cite{Cai2009}. They proposed to solve for the whole original solution $\phi$ in $\Omega^+$ instead of the reaction component $\phi_{RF}$ only. Then to make the required adjustments to fit this regularization approach with finite difference and finite element methods, Zhao and Geng \cite{Geng2017a} proposed a new interface problem with discontinuous flux jumps for the regularized potential $\tilde \phi$. In particular they defined it as 
\begin{eqnarray}
	\tilde{ \phi} &=& \begin{cases}
	\phi_{RF} \text{ in } \Omega^-\\
	\phi_C + \phi_{RF} \text{ in } \Omega^+\\
	\end{cases}.
\end{eqnarray}
 The jump conditions for $\tilde\phi $ were derived from the definition $\phi=\phi_C+\phi_{RF}$ as
\begin{eqnarray}
\phi^+=\phi^-_{RF}+\phi^-_C \text{  and  } \epsilon^+ \frac{\partial \phi^+}{\partial n}=\epsilon^- \frac{\partial \phi^-_{RF}}{\partial n}+\epsilon^- \frac{\partial \phi^-_C}{\partial n} \text{ on } \Gamma.
\end{eqnarray}
Hence, the regularized PBE (RPBE) with the corresponding interface and the boundary conditions takes the following form: 
\begin{eqnarray}
	-\nabla.(\epsilon^- \nabla \tilde{ \phi}) &=& 0 \text{ in } \Omega^-,\label{phitilde1}\\ 
	-\nabla.(\epsilon^+ \nabla \tilde{ \phi}) +\bar\kappa^2 \sinh(\tilde{ \phi})&=& 0 \text{ in } \Omega^+,\label{phitilde2}\\
	\left[\tilde{ \phi}\right] &=& G \text{ on } \Gamma, \label{phitilde3}\\ 
	\left[\epsilon\frac{\partial \tilde{ \phi}}{\partial n}\right]&=& \epsilon^-  \frac{\partial G}{\partial n} \text{ on } \Gamma,\label{phitilde4}\\
	\tilde{\phi} &=& \phi_b \text{ on } \partial \Omega. \label{phitilde5}
\end{eqnarray}	
Now we can summarize equation (\ref{phitilde1}) and equation (\ref{phitilde2}) as
\begin{equation}
	-\nabla . (\epsilon \nabla \tilde{ \phi}) +\bar\kappa^2 \sinh(\tilde{ \phi})=0 \text{ in } \Omega^-\cup\Omega^+.\label{eq:RPBE}
\end{equation}
 Zhao and Geng \cite{Geng2017a} proposed to numerically solve the regularized PB interface problem (RPBE) given in (\ref{phitilde1})-(\ref{phitilde5}) for $\tilde \phi$ and recovered the original solution as $\phi= \tilde\phi $ in $\Omega^+$ and $\phi = \tilde \phi + G $ in $\Omega^-$. Here the Green's function $G$ can be calculated analytically from  (\ref{eq:green}). We note that in  jump conditions (\ref{phitilde3}) and (\ref{phitilde4}) both the solution and its flux are discontinuous. 
%%%%%%%%%%%%%%%%%%%%%%%%%%%%%%%%%%%%%%%%%%%%%%%%%%%%%%%%%%%%%%%%%%%%%%%%%%%%%
\section{Pseudo-transient approach for the TRPBE}

As the ADI method in Section \ref{sec:adi-method}, a pseudo-time derivative has been introduced in the RPBE to solve the PBE as,  

%\begin{eqnarray}
%	 	\frac{\partial u}{\partial t} &=& \nabla.(\epsilon \nabla u)-\bar\kappa^2 \sinh(u) \text{ in } \Omega^- \cup \Omega^+ \label{eq:TRPBE} \\
%	 	  \left[u\right]&=& G,\text{ and } \left[\epsilon\frac{\partial u}{\partial n}\right]= \epsilon^-  \frac{\partial G}{\partial n} \text{ on } \Gamma\\ \label{jump}
%	 	  u&=&\phi_b \text{ on } \partial \Omega
%\end{eqnarray}
\begin{eqnarray}
	 	\frac{\partial \tilde\phi(\textbf{r},t)}{\partial t} &=& \nabla.(\epsilon({\bf r}) \nabla \tilde\phi(\textbf{r},t))-\bar\kappa^2({\bf r}) \sinh(\tilde\phi(\textbf{r},t)) \text{ in } \Omega^- \cup \Omega^+ \label{eq:TRPBE}. %\\
	 	  %\left[u\right]&=& G,\text{ and } \left[\epsilon\frac{\partial u}{\partial n}\right]= \epsilon^-  \frac{\partial G}{\partial n} \text{ on } \Gamma\\ \label{jump}
	 	%  u&=&\phi_b \text{ on } \partial \Omega
\end{eqnarray}
Here the time independent RPBE in (\ref{eq:RPBE}) has been converted into a time dependent regularized PBE (TRPBE) in (\ref{eq:TRPBE}). Unlike TPBE in (\ref{eq:tpbe}), TRPBE does not have any singular term $\rho({\bf r})$.   

As the initial condition, we used the electrostatic potential solved from a linearized PBE \cite{Zhao2011} or trivially $\tilde \phi= 0$. Then we numerically integrate (\ref{eq:TRPBE}) for a sufficiently long period to get the steady state solution as the solution of the original regularized PBE (\ref{eq:RPBE}). Here the sign on the right hand side of equation (\ref{eq:TRPBE}) has been considered as the reverse of the equation(\ref{eq:RPBE}) to ensure the numerical stability. 

However, there are still some challenges left for the numerical integration of the TRPBE (\ref{eq:TRPBE}) because of the requirement of long time integration, when explicit time stepping methods are usually not efficient \cite{Sayyed-Ahmad2004,Shestakov2002,Zhao2011,zhao_operator_2014}. Hence we employ a semi-implicit time splitting method \cite{Sayyed-Ahmad2004,Shestakov2002}, which have been commonly used to solve the TPBE (\ref{eq:tpbe}) in the literature. 

Let us consider a uniform mesh with a grid spacing $h$ in all $x,y$ and $z$ directions having $N_x,N_y$ and $N_z$ as the number of the grid points in each direction. We assume the vector ${{\bf u}^n = \tilde \phi^n_{ijk}}$ for $i=1,...., N_x,j=1,...,$ and $N_y,k=1,...N_z$ denote all the nodal values of $\tilde \phi$ at the time level $t_n$. We use two stages to update ${\bf u^n}=\tilde \phi^n$ at time level $t_n$ to ${\bf u^{n+1}}=\tilde \phi^{n+1}$ at time level $t_{n+1}=t_n+ \Delta t $. In these two stages at each time step we develop several types of operator splitting schemes for updating ${\bf u}^n$. 
  %%%%%%%%%%%%%%%%%%%%%%%%%%%%%%%%%%%%%%%%%%%%%%%%%%%%%%%%%%%%%%%%%%%%%%%%%%%%%
\section{Operator Splitting for the GFM-ADI method}
%\section{ADI method with Implicit-Euler method (GFM-ADI)}
\label{sec:GFM-ADI}

In this scheme at the first stage, TRPBE in  (\ref{eq:TRPBE}) will be solved by a first order time splitting into the two following equations similar to the ADI method discussed in Section (\ref{sec:adi-method}): 
\begin{eqnarray}
  \frac{\partial w}{\partial t}&=& -\bar\kappa^2 \sinh(w) \text{ with } {\bf w}^n={\bf u}^n\text{ and } t \in \left[t_n,t_{n+1}\right],\label{eq:non_linear_ADI}\\
 \frac{\partial v}{\partial t}&=&  \nabla . (\epsilon\nabla v) \text{ with } {\bf v}^n={\bf w}^{n+1}\text{ and } t \in \left[t_n,t_{n+1}\right].	 \label{eq:diffusion_ADI}
\end{eqnarray}  
For the first equation, (\ref{eq:non_linear_ADI}), we will use the analytical solution ${\bf w}^{n+1}=F({\bf w}^n,\Delta t)$ defined in (\ref{eq:anal_sol}). Then for the temporal discretization of the equation (\ref{eq:diffusion_ADI}) we use the Backward-Euler integration in time to get 
\begin{equation}
	v_{i,j,k}^{n+1} =v_{i,j,k}^{n}+\Delta t \left(\delta_x^2+\delta_y^2+\delta_z^2\right)v_{i,j,k}^{n+1}. \label{eq:imp-eu}
\end{equation}	
We cannot use the definition of $\Delta_x^2,\Delta_y^2$ and $\Delta_z^2$ in (\ref {eq:dif_opx_adi1}) for $\delta_x^2,\delta_y^2$ and $\delta_z^2$, since both the potential function $\tilde \phi$ and its flux are discontinuous on the interface $\Gamma$ as in equations (\ref{phitilde3}) and (\ref{phitilde4}). So we define $\delta_x^2,\delta_y^2$ and $\delta_z^2$ as a central finite difference operators in (\ref{dif_opz}) in the $x,y$ and $z$ directions, respectively with 
\begin{equation}
	\epsilon_{i,j,k}= \displaystyle\begin{cases}
	\epsilon^- \text{ if } x_{i,j,k}\in\Omega^- \cup \Gamma \\
	\epsilon^+ \text{ if } x_{i,j,k}\in\Omega^+
	\end{cases},
\end{equation}
as
 \begin{eqnarray}
 \begin{aligned}
	\delta_x^2\left(v_{i,j,k}^n\right)&= \frac{\epsilon_{i,j,k}}{h^2} \left(v_{i-1,j,k}^n-2v_{i,j,k}^n+v_{i+1,j,k}^n\right), \\ \label{dif_opx}
	\delta_y^2\left(v_{i,j,k}^n\right)&= \frac{\epsilon_{i,j,k}}{h^2} \left(v_{i,j-1,k}^n-2v_{i,j,k}^n+v_{i,j+1,k}^n\right),\\ %\label{dif_opy}
	\delta_z^2\left(v_{i,j,k}^n\right)&= \frac{\epsilon_{i,j,k}}{h^2} \left(v_{i,j,k-1}^n-2v_{i,j,k}^n+v_{i,j,k+1}^n\right). \label{dif_opz}
\end{aligned}
\end{eqnarray}
Even though these three point stencils for $\delta_x^2,\delta_y^2$ and $\delta_z^2$ give a huge advantage for the regular grid points, it posses a difficulty for the points adjacent to the interface. We will discuss more about this difficulty later in Chapter \ref{chap:new_GFM}.
  
Now for the second stage of the operator splitting, a first order Douglas-Rachford type ADI scheme is used to decompose the diffusion equation, (\ref{eq:imp-eu}), in $x,y$ and $z$ directions as  
\begin{eqnarray}
%\begin{aligned}
		\left(1- \Delta t \delta_x^2\right)v_{i,j,k}^{*}&=&\left[1+ \Delta t \left(\delta_y^2+\delta_z^2 \right)\right]v_{i,j,k}^{n},\label{GFM-ADI}\\
		\left(1-\Delta t \delta_y^2\right)v_{i,j,k}^{**}&=&v_{i,j,k}^{*}- \Delta t \delta_y^2\left(v_{i,j,k}^{n}\right),\label{GFM-ADI2}\\
		\left(1- \Delta t \delta_z^2\right)v_{i,j,k}^{n+1}&=&v_{i,j,k}^{**}- \Delta t \delta_z^2\left(v_{i,j,k}^{n}\right).\label{GFM-ADI3}
%\end{aligned}		\label{1dadi}
\end{eqnarray} 
where $v^*$ and $v^{**}$ are two intermediate values to create three tridiagonal one-dimensional systems. Here, the three dimensional linear algebraic system in equation (\ref{eq:imp-eu}) has been decomposed into several one dimensional linear algebraic systems in (\ref{GFM-ADI}), (\ref{GFM-ADI2}) and (\ref{GFM-ADI3}). The finite difference matrix for each one these linear equations has a tridiagonal structure. These three tridiagonal systems are much more efficient to solve because of the tridiagonal symmetry than one non-structured system (\ref{eq:imp-eu}). Then by eliminating $v^*_{i,j,k}$ and $v^{**}_{i,j,k}$ and solving for $v^{n+1}_{i,j,k}$ in (\ref{eq:imp-eu}) we get,
\begin{eqnarray}
\begin{aligned}
	v^{n+1}_{i,j,k}&=v^n_{i,j,k}+ \Delta t \left(\delta_x^2+\delta_y^2+\delta_z^2\right) v^{n+1}_{i,j,k} -\Delta t^2 \left(\delta_x^2\delta_y^2+\delta_x^2\delta_y^2+\delta_z^2\delta_y^2\right)(v^{n+1}_{i,j,k}-v^{n}_{i,j,k})\\
	&+ \Delta t^3 \delta_x^2\delta_y^2\delta_z^2(v^{n+1}_{i,j,k}-v^{n}_{i,j,k}). \label{adi_tailor}
	\end{aligned}
\end{eqnarray}
Hence the Douglas-Rachford scheme (\ref{eq:imp-eu}) is a higher order perturbation of the Implicit-Euler method. Since both (\ref{eq:non_linear_ADI}) and (\ref{eq:diffusion_ADI}) are first order in time this proposed GFM-ADI method is of first order accuracy in time. For the boundary conditions we use the same Dirichlet boundary boundary values from equation (\ref{bd_cond}) for $v$, $v^*$ and  $v^{**}$ for $u$. The entire time integration here is fully implicit. 


%%%%%%%%%%%%%%%%%%%%%%%%%%%%%%%%%%%%%%%%%%%%%%%%%%%%%%%%%%%%%%%%%%%%%%%%%%%%%
\section{Operator Splitting for the GFM-LODCN method}
%\section{LOD with Crank-Nicolson method(GFM-LODCN)}\label{GFM-LODCN}
\label{sec:GFM-LODCN}
In this operator splitting scheme at each time step from $t_n$ to $t_{n+1}$, the TRPBE (\ref{eq:TRPBE}) is splitted into the following three equations by a second order time splitting method \cite{Yu2005}:
\begin{eqnarray}
  \frac{\partial w}{\partial t}&=& -\frac{1}{2}\bar\kappa^2 \sinh(w) \text{ with } {\bf w}^n={\bf u}^n\text{ and } t \in \left[t_n,t_{n+1}\right],\label{non_linear1_LOD-CN}\\
 \frac{\partial v}{\partial t}&=&  \nabla . (\epsilon\nabla v) \text{    with } {\bf v}^n={\bf w}^{n+1}\text{ and } t \in \left[t_n,t_{n+1}\right],	 \label{diffusion_LOD-CN}\\
 \frac{\partial \tilde{w}}{\partial t}&=& -\frac{1}{2}\bar\kappa^2 \sinh( \tilde{w}) \text{ with } \tilde{\bf w}^n={\bf v}^{n+1}\text{ and } t \in \left[t_n,t_{n+1}\right].\label{non_linear2_LOD-CN}
\end{eqnarray}
We then have ${\bf u}^{n+1}=\tilde{\bf w}^{n+1}$. Similar to the GFM-ADI method in Section (\ref{sec:GFM-ADI}), at the first stage we use an analytical integration for the nonlinear equations, (\ref{non_linear1_LOD-CN}) and (\ref{non_linear2_LOD-CN}). Symbolically, we have ${\bf w}^{n+1}=F({\bf w}^n,\frac{\Delta t}{2})$ and $\tilde{\bf w}^{n+1}=F(\tilde{\bf w}^n,\frac{\Delta t}{2})$, where $F$ is defined in equation (\ref{eq:anal_sol}).

Then for the second stage, we propose another multiplicative operator splitting scheme, called the Locally One Dimensional (LOD) scheme to solve the diffusion equation (\ref{diffusion_LOD-CN}). These types of fractional step methods were first developed by Russian mathematicians \cite{Yakonov_1963,Yanenko_1963,Yanenko_1967}. The discretization of equation (\ref{diffusion_LOD-CN}) using Crank-Nicolson integration in time and central differencing in space results in, 
\begin{eqnarray}
	\left(1-\frac{\Delta t}{2}(\delta^2_x+\delta^2_y+\delta^2_z)\right)v_{i,j,k}^{n+1}=\left(1+\frac{\Delta t}{2}(\delta^2_x+\delta^2_y+\delta^2_z)\right)v_{i,j,k}^n,
\end{eqnarray}
which can be decomposed into $x,y$ and $z$ directions to give the LOD scheme for (\ref{diffusion_LOD-CN}) as
\begin{eqnarray}
	\left(1-\frac{\Delta t}{2}\delta_x^2\right)v^*_{i,j,k}&=\left(1+\frac{\Delta t}{2}\delta_x^2\right)v^n_{i,j,k},\nonumber\\
	\left(1-\frac{\Delta t}{2}\delta_y^2\right)v^{**}_{i,j,k}&=\left(1+\frac{\Delta t}{2}\delta_y^2\right)v^*_{i,j,k},\nonumber\\
	\left(1-\frac{\Delta t}{2}\delta_z^2\right)v^{n+1}_{i,j,k}&=\left(1+\frac{\Delta t}{2}\delta_z^2\right)v^{**}_{i,j,k}.\nonumber\label{LODCN_eq}
\end{eqnarray}

Similar to the GFM-ADI method described in Section \ref{sec:GFM-ADI}, the tridiagonal systems in (\ref{LODCN_eq}) can be efficiently solved by the Thomas algorithm. 

%%%%%%%%%%%%%%%%%%%%%%%%%%%%%%%%%%%%%%%%%%%%%%%%%%%%%%%%%%%%%%%%%%%%%%%%%%%%%

\section{Operator Splitting for the GFM-LODIE method}
%\section{LOD with Implicit-Euler method (GFM-LODIE)}

For this method at each time step from $t_n$ to $t_{n+1}$, the TRPBE (\ref{eq:TRPBE}) is  solved in two stages similar to the GFM-ADI scheme described in Section \ref{sec:GFM-ADI}. 
The first stage is exactly the same to the GFM-ADI method. The first order time splitting is used for 
 TRPBE (\ref{eq:TRPBE}) to generate   equations (\ref{eq:non_linear_ADI}) and (\ref{eq:diffusion_ADI}). Equation (\ref{eq:non_linear_ADI}) is solved analytically by the function ${\bf w}^{n+1}=F({\bf w}^n,\Delta t)$ defined in equation (\ref{eq:anal_sol}).
 
 Then for the second stage, we apply the LOD scheme described in Section \ref{sec:GFM-LODCN} to decompose equation (\ref{eq:imp-eu}) in $x,y$ and $z$ directions:
 \begin{eqnarray}
		(1-\Delta t \delta_x^2)v^*_{i,j,k} &= v^n_{i,j,k},\nonumber\\
		(1-\Delta t \delta_y^2)v^{**}_{i,j,k} &= v^*_{i,j,k},\\
		(1-\Delta t \delta_z^2)v^{n+1}_{i,j,k} &= v^{**}_{i,j,k},\nonumber
\end{eqnarray}	
which are solved again individually by the Thomas Algorithm used in the previous sections. 

%%%%%%%%%%%%%%%%%%%%%%%%%%%%%%%%%%%%%%%%%%%%%%%%%%%%%%%%%%%%%%%%%%%%%%%%%%%%%%%%%%%%%
\section{Solvation energy from Electrostatic potential}

The solvation energy can be defined as the energy released when the solute from the free space is dissolved in the solvent and it can be expressed in terms of the electrostatic free energy $\Delta G_{\rm ele}$ and the coulomb energy $E_{\rm cou}$ as
\begin{equation}
	\Delta G_{\rm ele}= E_{\rm cou}+ E_{\rm sol},\label{eq_Gele}
\end{equation}
where
\begin{equation}	
	E_{\rm cou} = \sum_{i=1}^{N_c}\sum_{j=1}^{N_c}\frac{q_iq_j}{\epsilon^-d_{i,j}}, i\neq j,\label{eq_Ecou}
\end{equation}
and $q_i$ and $q_j$ are the charges at the center of the atoms, and $d_{i,j}$ is the distance between the $i$-th and $j$-th atom. 

Sharp and Honig \cite{Sharp_Honig} describe the calculation of the electrostatic free energy by using 
\begin{equation}
\Delta G_{\rm ele}=\int_{\mathbb{R}^3}\left(\phi\rho+\Delta\Pi-\frac{1}{2}\epsilon|{\bf E}|^2\right)d{\bf r},
\end{equation}
where $\phi$ is the electrostatic potential, $\rho$ is the fixed charge density represented as a smeared surface charge or as a collection of point charges, $\Delta\Pi$ is  the excess osmotic pressure of the mobile ion cloud, and $\frac{1}{2}\epsilon|{\bf E}|^2$ is the electrostatic stress.

For the simplification of the numerical validations of our proposed schemes in the Chapter \ref{chap:num_vald} we omitted the energy components related to the mobile ion pressure and the electrostatic stress to report the solvation energy $E_{\rm sol}$ as  
\begin{equation}
	E_{\rm sol}=\Delta G_{\rm ele}-E_{\rm cou}\approx\sum\limits_{i=1}^{N_c}q_i\phi_{RF} ({\bf r}_i).\label{eq:solvation}
\end{equation}
Here the contribution of the mobile ion pressure and the electrostatic stress to the whole calculation is really small and computationally more challenging. The readers can refer to \cite{GENG_WEI2011,Gilson,Sharp_Honig} for more details.	 
