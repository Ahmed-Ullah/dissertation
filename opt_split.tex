Let us consider a uniform mesh with a grid spacing $h$ in all $x,y$ and $z$ directions having $N_x,N_y$ and $N_z$ as the number of the grid points in each direction. We assume the vector $U^n={\tilde\phi^n_{ijk}}$ for $i=1,...., N_x,j=1,...,N_y,k=1,...N_z$ denote all the nodal values of $\tilde\phi$ at the time level $t_n$. We will develop several schemes for updating $U^n$ at time level $t_n$ to $U^{n+1}$ at time level $t_{n+1}=t_n+ \Delta t $. In this time dependent RPB we will consider only $U$ as time dependent, while all other functions in (\ref{tdrpb}) i.e. $\epsilon$ and $\bar\kappa^2 $ are time independent. 
%*************************************************************************************
\section{Implicit-Euler method with Alternating Direction Implicit scheme (GFM-ADI)}\label{GFM-ADI_sec}

In this scheme at each time step from $t_n$ to $t_{n+1}$, the time dependent equation TDRPB in  (\ref{tdrpb}) will be solved by a first order time splitting in two stages, 
\begin{eqnarray}
  \frac{\partial w}{\partial t}&=& -\bar\kappa^2 \sinh(w) \text{ with } W^n=U^n\text{ and } t \in \left[t_n,t_{n+1}\right]\label{non_linear_ADI}\\
 \frac{\partial v}{\partial t}&=&  \nabla . (\epsilon\nabla v) \text{ with } V^n=W^{n+1}\text{ and } t \in \left[t_n,t_{n+1}\right]	 \label{diffusion_ADI}
\end{eqnarray}  
Altogether we have $U^{n+1}=V^{n+1}$. And for $ \frac{\partial w}{\partial t}= -\bar\kappa^2 \sinh(w)$ we have the analytical solution as, 	
	\begin{eqnarray}	
W^{n+1}= \ln \left( \frac{\cosh(\frac{1}{2}\bar\kappa^2\Delta t)+\exp(-W^n)\sinh(\frac{1}{2}\bar\kappa^2\Delta t)}{\exp(-W^n)\cosh(\frac{1}{2}\bar\kappa^2\Delta t)+\sinh(\frac{1}{2}\bar\kappa^2\Delta t)}\right)\label{anal_sol}
\end{eqnarray}
Which will help us to avoid to the difficulty due to the non-linear term with $\sinh(.)$ in (\ref{pbe}). Here the right hand side of equation (\ref{anal_sol}) is just a function of $W$ and $\Delta t$ which allows us to rewrite the equation $(\ref{anal_sol})$ as $W^{n+1}=F(W^n,\Delta t)$ to facilitate the following discussion. 
For the temporal discretization of the equation (\ref{diffusion_ADI}) we have used Backward-Euler integration in time to get, 
\begin{eqnarray}
	v_{i,j,k}^{n+1} &=v_{i,j,k}^{n}+\Delta t \left(\delta_x^2+\delta_y^2+\delta_z^2\right)v_{i,j,k}^{n+1} \label{imp-eu}
\end{eqnarray}	

 Where $\delta_x^2,\delta_y^2$ and $\delta_z^2$ are the central finite difference operators defined in (\ref{dif_opz}) for the $x,y$ and $z$ directions, respectively with $\epsilon_{i,j,k}= \displaystyle\begin{cases}
	\epsilon^- \text{ if } x_{i,j,k}\in\Omega^- \cup \Gamma \\
	\epsilon^+ \text{ if } x_{i,j,k}\in\Omega^+\\
	\end{cases}$.

 \begin{eqnarray}
 \begin{aligned}
	\delta_x^2\left(v_{i,j,k}^n\right)&= \frac{\epsilon_{i,j,k}}{h^2} \left(v_{i-1,j,k}^n-2v_{i,j,k}^n+v_{i+1,j,k}^n\right) \\ \label{dif_opx}
	\delta_y^2\left(v_{i,j,k}^n\right)&= \frac{\epsilon_{i,j,k}}{h^2} \left(v_{i,j-1,k}^n-2v_{i,j,k}^n+v_{i,j+1,k}^n\right)\\ %\label{dif_opy}
	\delta_z^2\left(v_{i,j,k}^n\right)&= \frac{\epsilon_{i,j,k}}{h^2} \left(v_{i,j,k-1}^n-2v_{i,j,k}^n+v_{i,j,k+1}^n\right) \label{dif_opz}
\end{aligned}
\end{eqnarray}

%\end{equation}
%on the points far away from the interface $\Gamma$ which are defined as regulars nodes.
But at the  points near the interface the central difference operators defined in (\ref{dif_opx})-(\ref{dif_opz}) will not be applicable. These points are defined as the irregular points where at-least one of its adjacent points is on the other side of the boundary. On the irregular points one of the three point stencils for $\delta_x^2,\delta_y^2$ and $\delta_z^2$  will be on the other side of the interface where the information about the function $v$ is not available. A non-standard finite difference formula is necessary on the irregular points to discretize $\delta_x^2,\delta_y^2$ and $\delta_z^2$.  In this regard a modified version of the Ghost Fluid Method has been proposed in Section \ref{new_gfm} using the fictitious points (or ghost points) and the jump conditions in (\ref{jump}) rigorously.

Then a first order Douglas-Rachford type ADI scheme has been used to decompose the diffusion equation (\ref{imp-eu}) in $x,y$ and $z$ directions as,  
\begin{eqnarray}
%\begin{aligned}
		\left(1- \Delta t \delta_x^2\right)v_{i,j,k}^{*}&=&\left[1+ \Delta t \left(\delta_y^2+\delta_z^2 \right)\right]v_{i,j,k}^{n}\label{GFM-ADI}\\
		\left(1-\Delta t \delta_y^2\right)v_{i,j,k}^{**}&=&v_{i,j,k}^{*}- \Delta t \delta_y^2\left(v_{i,j,k}^{n}\right)\label{GFM-ADI2}\\
		\left(1- \Delta t \delta_z^2\right)v_{i,j,k}^{n+1}&=&v_{i,j,k}^{**}- \Delta t \delta_z^2\left(v_{i,j,k}^{n}\right)\label{GFM-ADI3}
%\end{aligned}		\label{1dadi}
\end{eqnarray} 
Where $v^*$ and $v^{**}$ are two intermediate values to create three tridiagonal one dimensional system. Here, the three dimensional linear algebraic system in equation (\ref{imp-eu}) has been decomposed into several one dimensional linear algebraic systems in (\ref{GFM-ADI}), (\ref{GFM-ADI2}) and (\ref{GFM-ADI3}).  Each one these equations has a tridiagonal structure. These three tridiagonal systems are much more efficient to solve than one non-structured system in (\ref{imp-eu}). Then by eliminating $v^*_{i,j,k}$ and $v^{**}_{i,j,k}$ and solving for $v^{n+1}_{i,j,k}$ in (\ref{imp-eu}) we get,
\begin{eqnarray}
\begin{aligned}
	v^{n+1}_{i,j,k}&=v^n_{i,j,k}+ \Delta t \left(\delta_x^2+\delta_y^2+\delta_z^2\right) v^{n+1}_{i,j,k} -\Delta t^2 \left(\delta_x^2\delta_y^2+\delta_x^2\delta_y^2+\delta_z^2\delta_y^2\right)(v^{n+1}_{i,j,k}-v^{n}_{i,j,k})\\
	&+ \Delta t^3 \delta_x^2\delta_y^2\delta_z^2(v^{n+1}_{i,j,k}-v^{n}_{i,j,k}) \label{adi_tailor}
	\end{aligned}
\end{eqnarray}
Hence the Douglas-Rachford scheme (\ref{imp-eu}) is a higher order perturbation of the Implicit-Euler method. Since both (\ref{non_linear_ADI}) and (\ref{diffusion_ADI}) are first order in time this proposed ADI-IE-GFM scheme is of first order accuracy in time. As the boundary conditions the same Dirichlet boundary boundary values are assumed for $v$, $v^*$ and  $v^{**}$ for $u$. The entire time integration here is fully implicit. 
%*************************************************************************************
\section{Crank-Nicolson method with Locally One Dimensional Scheme(GFM-LODCN)}\label{GFM-LODCN}
In this scheme at each time step from $t_n$ to $t_{n+1}$, the time dependent non-linear equation TDPBE will be solved by a second order time splitting methods in three stages \cite{Yu2005},
\begin{eqnarray}
  \frac{\partial w}{\partial t}&=& -\frac{1}{2}\bar\kappa^2 \sinh(w) \text{ with } W^n=U^n\text{ and } t \in \left[t_n,t_{n+1}\right]\label{non_linear1_LOD-CN}\\
 \frac{\partial v}{\partial t}&=&  \nabla . (\epsilon\nabla v) \text{    with } V^n=W^{n+1}\text{ and } t \in \left[t_n,t_{n+1}\right]	 \label{diffusion_LOD-CN}\\
 \frac{\partial \tilde{w}}{\partial t}&=& -\frac{1}{2}\bar\kappa^2 \sinh( \tilde{w}) \text{ with } \tilde{W}^n=V^{n+1}\text{ and } t \in \left[t_n,t_{n+1}\right]\label{non_linear2_LOD-CN}
\end{eqnarray}
We then have $U^{n+1}=\tilde{W}^{n+1}$. As in the subsection \ref{GFM-ADI_sec} we will use an analytical integration in the first and the last stage of this scheme. Symbolically, we have $W^{n+1}=F(W^n,\frac{\Delta t}{2})$ and $\tilde{W}^{n+1}=F(\tilde{W}^n,\frac{\Delta t}{2})$, where $F$ is defined as in Eq.(\ref{anal_sol}).\\
Then we have proposed another multiplicative operator splitting scheme called as Locally One Dimensional (LOD) scheme to solve the diffusion equation (\ref{diffusion_LOD-CN}). This types of fractional step methods were first developed by Russian mathematicians in \cite{Yakonov_1963}, \cite{Yanenko_1963}, \cite{Yanenko_1967}. The discretization of (\ref{diffusion_LOD-CN}) using Crank-Nicolson integration in time and central differencing in space results in, 
\begin{eqnarray}
	\left(1-\frac{\Delta t}{2}(\delta^2_x+\delta^2_y+\delta^2_z)\right)v_{i,j,k}^{n+1}=\left(1+\frac{\Delta t}{2}(\delta^2_x+\delta^2_y+\delta^2_z)\right)v_{i,j,k}^n
\end{eqnarray}
Which can be decomposed in to $x,y$ and $z$ directions to give a LOD scheme for (\ref{diffusion_LOD-CN}),
\begin{eqnarray}
	\begin{aligned}
	\left(1-\frac{\Delta t}{2}\delta_x^2\right)v^*_{i,j,k}&=\left(1+\frac{\Delta t}{2}\delta_x^2\right)v^n_{i,j,k}\\
	\left(1-\frac{\Delta t}{2}\delta_y^2\right)v^{**}_{i,j,k}&=\left(1+\frac{\Delta t}{2}\delta_y^2\right)v^*_{i,j,k}\\
	\left(1-\frac{\Delta t}{2}\delta_z^2\right)v^{n+1}_{i,j,k}&=\left(1+\frac{\Delta t}{2}\delta_z^2\right)v^{**}_{i,j,k}\label{LODCN_eq}
	\end{aligned}
\end{eqnarray}

Similarly as the GFM-ADI scheme described in subsection \ref{GFM-ADI_sec} the tridiagonal systems in (\ref{LODCN_eq}) can be efficiently solved by the Thomas algorithm. 

%*************************************************************************************

\section{Implicit-Euler method with Locally One Dimensional Scheme(GFM-LODIE)}

For this scheme at each time step from $t_n$ to $t_{n+1}$, the time dependent equation TDRPB in  (\ref{tdrpb}) will be solved by the first order time splitting  similar to the GFM-ADI scheme described in the subsection \ref{GFM-ADI_sec} in two stages. To follow the same steps in the first stage Eq. (\ref{non_linear_ADI}) will be solved analytically by the function $W^{n+1}=F(W^n,\Delta t)$ in Eq. (\ref{anal_sol}) and in the second stage for the temporal discretization of the Eq.  (\ref{diffusion_ADI}) we have used Implicit Euler method to get Eq.(\ref{imp-eu}). 
 
 Then we have applied the LOD scheme described in the subsection \ref{GFM-LODCN} to decompose Eq. (\ref{imp-eu})  in to $x,y$ and $z$ direction as, 
 
\begin{eqnarray}
\begin{aligned}
		(1-\Delta t \delta_x^2)v^*_{i,j,k} &= v^n_{i,j,k}\\
		(1-\Delta t \delta_y^2)v^{**}_{i,j,k} &= v^*_{i,j,k}\\
		(1-\Delta t \delta_z^2)v^{n+1}_{i,j,k} &= v^{**}_{i,j,k}\\
	\end{aligned}
\end{eqnarray}	
Which will solved individually by the Thomas Algorithm like the previous subsections. 
% At each time step from $t_n$ to $t_{n+1}$ the diffusion equation (\ref{non_linear_ADI}) and (\ref{diffusion_ADI}) in fi
%
%\begin{eqnarray}
%\begin{aligned}
% \frac{\partial w}{\partial t}=& -\bar\kappa^2 \sinh(w) \text{ with } W^n=U^n\text{ and } t \in \left[t_n,t_{n+1}\right] \\
%   \frac{\partial v}{\partial t}=& \frac{\partial}{\partial x}\left(\epsilon \frac{\partial v}{\partial x}\right ), \text{ with } V^n=W^{n+1}\text{ and } t \in \left[t_n,t_{n+1}\right]\\	
%     \frac{\partial p}{\partial t}=& \frac{\partial}{\partial y}\left(\epsilon \frac{\partial p}{\partial y}\right ), \text{ with } P^n=V^{n+1}\text{ and } t \in \left[t_n,t_{n+1}\right]\\
%      \frac{\partial q}{\partial t}=& \frac{\partial}{\partial z}\left(\epsilon \frac{\partial q}{\partial z}\right ), \text{ with } Q^n=P^{n+1}\text{ and } t \in \left[t_n,t_{n+1}\right]	
%%       \frac{\partial s}{\partial t}&=& 0, \text{ with } S^n=Q^{n+1}\text{ and } t \in \left[t_n,t_{n+1}\right]
%\end{aligned}	
%\end{eqnarray}
%And finally $U^{n+1}=P^{n+1}$. Which will give us the following systems to solve,
%	\begin{eqnarray}
%	\begin{aligned}
%		w_{i,j,k} =& \ln \left[ \frac{\cosh(\frac{1}{2}\bar\kappa^2\Delta t)+\exp(-u^n_{i,j,k})\sinh(\frac{1}{2}\bar\kappa^2\Delta t)}{\exp(-u^n_{i,j,k})\cosh(\frac{1}{2}\bar\kappa^2\Delta t)+\sinh(\frac{1}{2}\bar\kappa^2\Delta t)}\right]\\
%		(1-\Delta t \delta_x^2)v_{i,j,k} =& w_{i,j,k}\\
%		(1-\Delta t \delta_y^2)p_{i,j,k} =& v_{i,j,k}\\
%		(1-\Delta t \delta_z^2)q_{i,j,k} =& p_{i,j,k}\\
%		u^{n+1}_{i,j,k} =& q_{i,j,k} \label{lod_split}
%		\end{aligned}
%	\end{eqnarray}	
%\subsection{Spatial Discritization}
%		We will approximate the the spatial derivatives in (\ref{dif2}), (\ref{1dadi}), (\ref{adi_tailor}) and (\ref{lod_split}) by the finite difference operators $\delta_{xx}, \delta_{yy}$ and $\delta_{zz}$ by the standard central finite difference formula as,  
%		\begin{eqnarray}
%			\begin{aligned}
%				\frac{\partial^2}{\partial x^2}\left(u_{i,j,k}^n\right)& \approx \delta_{xx}\left(u_{i,j,k}^n\right):= \frac{1}{\Delta x^2} \left(u_{i-1,j,k}^n-2u_{i,j,k}^n+u_{i+1,j,k}^n\right)\\
%				\frac{\partial^2}{\partial y^2}\left(u_{i,j,k}^n\right)& \approx \delta_{yy}\left(u_{i,j,k}^n\right):= \frac{1}{\Delta y^2} \left(u_{i,j-1,k}^n-2u_{i,j,k}^n+u_{i,j+1,k}^n\right)\\
%				\frac{\partial^2}{\partial z^2}\left(u_{i,j,k}^n\right)& \approx \delta_{zz}\left(u_{i,j,k}^n\right):= \frac{1}{\Delta z^2} \left(u_{i,j,k-1}^n-2u_{i,j,k}^n+u_{i,j,k+1}^n\right)\label{space_discrit}
%			\end{aligned}
%		\end{eqnarray}
%	 
