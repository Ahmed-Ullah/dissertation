Solvated biomolecules and the electrostatic interaction with its solvent is critical to study various important biological process such as protein-drug binding site analysis, DNA recognition, protein folding and protein ligand bonding. In the past few decades with the development of numerical methods and computational techniques analyzing the functions and dynamics of bimolecular solvation is more practical and effective. However, imitating these interactions are still computationally expensive with biological significance. Here we are considering the Poisson-Boltzmann Equation (PBE)  to describe the electrostatic potential generated by a low dielectric medium inside a protein molecule with embedded atomic charges solvated in a high dielectric medium with dissolved ions. The analytical solution of PBE is only available for some simple geometry such as a sphere or a cylinder. Efficiency and accuracy becomes a critical issue to solve PBE for biophysical models with complex geometry and a macro-molecule containing tens of thousands to millions of atoms (having partial charges). 

 In our mathematical model for the electrostatic analysis, PBE is a nonlinear elliptic equation on multiple domains with discontinuous dielectric coefficients separated  by the solute-solvent interface or molecular surface. The difficulties with solving PBE arises from non-linearity, discontinuous dielectric coefficients, non-smoothness of the solution and singularities in the source term due to the atomic charges. Effects of non-linearity becomes significant with strong ionic presence~\cite{Wilson2016}. 


For the non-linearity, two different approaches have been developed in the literature. The usual approach is to discretize the nonlinear PBE into an algebraic system using finite difference or finite element methods and then solve it by a nonlinear algebraic system such as (1) nonlinear relaxation method \cite{Im1998}, \cite{Rocchia2001}, (2) nonlinear conjugate gradient method~\cite{Luty1992} or (3) inexact Newton method \cite{Holst1995}. The other approach has been introduced recently and its based on the  pseudo-transient continuation approach \cite{Shestakov2002,Sayyed-Ahmad2004,Zhao2011}. This approach converts time independent nonlinear PBE into a time dependent form by introducing pseudo-time derivative. The solution to the original boundary problem is then retrieved from the steady state solution of the time dependent PBE. The main advantage of introducing pseudo-time derivative is to be able to split time dependent PBE into linear and nonlinear subsystems to circumvent the blow up and overflow problem due to exponentially large term involving hyperbolic sine function. 

To convert this three dimensional(3D) PBE into a set of multiple independent one-dimensional(1D) systems, the alternating direction implicit (ADI) methods in \cite{Fogolari2002,Geng2015,Geng2013_tree,Geng2013_Fully} the locally one dimensional method (LOD) in \cite{Wilson2016} have been used in the literature. Specially in \cite{Geng2013_Fully} Douglas-Rachford ADI scheme has been used to split the linear subsystem with the 3D laplacian operator into three more sub systems with one dimensional 2nd order derivatives. Altogether this method remains 2nd order of accuracy in time but conditionally stable. Later in \cite{Wilson2016} the LOD method appeared as an unconditionally stable method with reduced accuracy compared to ADI methods in \cite{Geng2013_Fully}. Even though these 1D subsystems produced by these ADI and LOD methods are tridiagonal and can be efficiently solved by using the Thomas Algorithm \cite{FD_PDE}, it lacks the treatment of the jump conditions at the interface which reduces the accuracy near the interface. Also the numerical error for these pseudo transient methods has been observed often to be dominated by the the singularity at the center of the atoms.

Even though the MIB solvers were pretty efficient to overcome the difficulty due to the jump conditions at the interface, the efficient treatment of charge singularity still remained challenging. It motivated many authors to develop different regularization methods in \cite{Cai2009,Chen2007, Geng2007,Holst2010,XIE2014,Zhou_1996,Geng2017a} to reduce the loss of accuracy due to the singularity. In these methods, the potential function has decomposed into a singular component plus one or two other components to break down the PBE into a system of partial differential equations (PDEs) containing a poisson equation with the singular term plus one or two other equations. Thus the singular component can be handled separately using the analytical solution for the poisson equation as Coulomb potentials and or Green's functions. So far these type of regularization methods have never been used with the pseudo transient methods like ADI or LOD. 

In an attempt to maintain the efficiency and stability of the ADI methods while restoring the accuracy to the second order near the interface, several interface schemes has been developed in \cite{Li2017,Li1999,Mayo1993,Liu2013}. Then as a continuation of this approach recently matched ADI (mADI) method was developed in \cite{Zhao2015} and \cite{Li2017} to combine the MIB method (for interface treatment) with ADI. But it was mainly focused on the parabolic equations and only 2D condition was considered for the numerical validation.

In this paper our goal was to develop a new approach to solve the PBE combining the regularization, the pseudo-transient continuation and the interface treatment so that both nonlinearity and singularity are treated. For the regularization we have chosen  the two component regularization used in \cite{Geng2017a}. But it changes the jump condition to be non zero which introduces the necessity of interface treatments. Otherwise the standard central difference becomes divergent. Then motivated by the mADI method we have introduced a modified version of the Ghost Fluid Method (GFM)  in \cite{Fedkiw1999} as the interface treatment. Compared to mADI, GFM is simpler to apply in a pseudo-continuation approach. Altogether GFM-ADI method improved the accuracy and efficiency of the ADI method to solve the non-liner PB equation. Generally it more robust than ADI method but still fails to converge in some special cases. Then to continue the search of a more stable method for our regularized pseudo continuation approach we replaced ADI scheme  by LOD method to propose GFM-LODCN and GFM-LODIE method. These other two methods are combining LOD  with Crank-Nicolson (CN) and Implicit Euler (IE) to discretize the pseudo time derivative. All of the three methods produced more accurate results efficiently than their predecessors. GFM-LODIE has found to be most  robust while GFM-ADI to be most accurate.