Solvated biomolecules and their electrostatic interaction with the surrounding solvent are critical to the studies of various important biological processes such as protein-drug binding site analysis, DNA recognition, protein folding and protein ligand bonding. In the past few decades with the development of numerical methods and computational powers, the electrostatic analysis of functions and dynamics of bimolecular solvation has become more practical and effective. However, imitating these interactions are still computationally expensive with biological significance. Here we are considering the Poisson-Boltzmann Equation (PBE)  to describe the electrostatic potential generated by a low dielectric medium inside a protein molecule with embedded atomic charges solvated in a high dielectric medium with dissolved ions. The analytical solution of PBE is only available for some simple geometry such as a sphere or a cylinder. Efficiency and accuracy are still critical issues in numerical solution of the PBE for biophysical models with complex geometry, especially for macromolecules  containing tens of thousands to millions of atoms. 

 In our mathematical model for the electrostatic analysis, PBE is a nonlinear elliptic equation on multiple domains with discontinuous dielectric coefficients separated  by the solute-solvent interface or molecular surface. The difficulties with solving PBE arises from nonlinearity, discontinuous dielectric coefficients, non-smoothness of the solution and singularities in the source term due to the atomic charges. Effects of nonlinearity becomes significant with strong ionic presence~\cite{Wilson2016}. 


For the nonlinearity, two different approaches have been developed in the literature. The usual approach is to discretize the nonlinear PBE into an algebraic system using finite difference or finite element methods and then solve it by a nonlinear algebraic system such as nonlinear relaxation method \cite{Im1998}, \cite{Rocchia2001}, nonlinear conjugate gradient method~\cite{Luty1992} or inexact Newton method \cite{Holst1995}. The other approach has been introduced recently based on the  pseudo-transient continuation idea \cite{Shestakov2002,Sayyed-Ahmad2004,Zhao2011}. This approach converts time independent nonlinear PBE into a time dependent form by introducing a pseudo-time derivative. The solution to the original boundary problem is then retrieved from the steady state solution of the time dependent PBE. The main advantage of introducing pseudo-time derivative is to be able to split time dependent PBE into linear and nonlinear subsystems to circumvent the blow up and overflow problem due to exponentially large term involving hyperbolic sine function. 

In pseudo-time methods, the time dependent PBE has to be solved until steady state. To maintain the efficiency, a large time increment for  $\Delta t$ is required. This is why the existing pseudo-time methods usually adopted an implicit scheme in time stepping. Moreover, to convert the three dimensional(3D) PBE into a set of multiple independent one-dimensional(1D) systems, the alternating direction implicit (ADI) methods in \cite{Geng2013_tree,Geng2013_Fully}, the locally one dimensional (LOD) method in \cite{Wilson2016} have been introduced in the literature. Specially in \cite{Geng2013_Fully} Douglas-Rachford ADI scheme has been used to split the linear subsystem with the 3D laplacian operator into three sub-systems with one dimensional 2nd order derivatives. Altogether this method has 1st order of accuracy in time but a quite severe stability condition,. Later in \cite{Wilson2016} the LOD method was introduced as an unconditionally stable method with reduced accuracy compared to ADI methods in \cite{Geng2013_Fully}. Even though these 1D subsystems produced by these ADI and LOD methods are tridiagonal and can be efficiently solved by using the Thomas Algorithm \cite{FD_PDE}, they lack the treatment of the jump conditions at the interface which reduces the spatial accuracy near the interface. Also the numerical error for these pseudo transient methods has been observed often to be dominated by the the singularity at the center of the atoms.

Besides strong nonlinearity, the numerical treatment of charge singularity is another challenge for the PBE. At atom centers, both charge source and potential solution blow up, and the conventional discretization is doomed to be inaccurate. This motivated many authors to develop different regularization methods in \cite{Cai2009,Chen2007, Geng2007,Holst2010,XIE2014,Zhou_1996,Geng2017a} to reduce the loss of accuracy due to the singularity. In these methods, the potential function is decomposed into a singular component plus one or two other components to break down the PBE into a system of partial differential equations (PDEs) containing a poisson equation with the singular term plus one or two other equations. Thus the singular component can be handled separately using the analytical solution for the poisson equation as Coulomb potentials or Green's functions. So far these type of regularization methods have never been used with the ADI or LOD type pseudo transient methods. 

The dielectric interface is also crucial in numerical discretization of the PBE as it defines the boundary for the solute and solvent regions. Across a geometrically complex dielectric interface, or molecular surface, the potential solution is continuous, but its normal derivative is discontinuous. For un-regularized PBE, the standard finite difference method is still convergent and  degenerated to first order convergence in space. However, the situation becomes worse for regularized PBE, because now both potential and its flux will be discontinuous across the interface. The standard finite difference solution will diverge in this case, if no interface treatment is imposed. For this reason, the regularized PBE is usually solved by some special interface schemes, such as matched interface and boundary (MIB) method \cite{Geng2007,Chen2011,Yu2007,ZHAO2004,ZHOU2006,ZHOU2006_high,YU2007_3D}. We note that regularization methods have been applied with the finite element type pseudo transient methods in \cite{DENG2018}, in which the interface jump conditions can be built in the variational formulations. However, these methods are usually inefficient by solving a large linear system iteratively at each time step. 

In an attempt to maintain the efficiency and stability of the ADI methods while restoring the accuracy to the second order near the interface, several interface schemes have been developed for solving the diffusion equationin \cite{Li1999,Liu2013}. Then as a continuation of this approach recently matched ADI (mADI) method was developed in \cite{Zhao2015} and \cite{Li2017} to combine the MIB method (for interface treatment) with ADI. But these ADI methods were mainly focused on the parabolic equations and have never been applied to the PBE. In fact, the mADI \cite{Zhao2015} could become cumbersome in treating complicated interface, like the molecular surface in protein studies. 

In this study our goal was to develop a new approach to solve the PBE combining the regularization, the pseudo-transient continuation and the interface treatment so that both nonlinearity and singularity are properly treated. For the regularization we have chosen  the two component regularization developed in \cite{Geng2017a} which is the simplest and most accurate regularization method. But it changes the jump condition to be non zero which introduces the necessity of interface treatments. Otherwise the standard central finite difference becomes divergent. Then motivated by the mADI method we have generalised the Ghost Fluid Method (GFM) developed in \cite{Fedkiw1999} as the interface treatment for the present study. Compared to mADI, GFM is simpler to apply in a pseudo-continuation approach. Altogether GFM-ADI method improved the accuracy and efficiency of the ADI method to solve the nonlinear PBE. Generally it is more robust than ADI method but still has a time stability constraint when the time step size is too huge. Then to continue the search of a more stable method for our regularized pseudo continuation approach we replaced ADI scheme by LOD formulation to propose GFM-LODCN and GFM-LODIE methods. These two methods are combining LOD  with Crank-Nicolson (CN) and Implicit Euler (IE) to discretize the pseudo time derivative. All of the three methods produced more accurate and efficient results  than their predecessors. In particular GFM-LODIE has found to be most  robust while GFM-ADI to be most accurate. 
%%%%%%%%%%%%%%%%%%%%%%%%%%%%%%%%%%%%%%%%%%%%%%%%%%%%%%%%%%%%%%%%%%%%%%%%%%%%%%%%%%%%%%%%%%%%%
\section{Outlines of this dissertation}
There are six main parts in this dissertation. The first part discusses the Protein data file preparations necessary for the numerical algorithm developed to calculate the electrostatic potential and the solvation energy by the PBE. The second part introduces the Poisson-Boltzmann model and discusses the ADI method \cite{Geng2013_Fully} to give an analytical background of the previous ADI methods, molecular surfaces and coding packages. The third part discusses the two component regularization and introduces three pseudo transient methods GFM-ADI, GFM-LODCN and GFM-LODIE.  The fourth part introduces the a new GFM method to incorporate with the proposed pseudo transient methods. The last part validates the proposed method for a benchmark problem and examines the application the newly proposed methods to the PBE model for the real proteins. Below is a breakdown of the following chapters in greater details:  
  

{\bf Chapter \ref{chap: protein_data}} starts with a description of the Protein Data Bank and its different file formats. A detail description of {\it .pdb} file is included with the process to convert them to get the {\it .pqr} file. The types of inputs like $x,y$ and $z$  coordinate, the Van der Walls radius and the charges for the numerical algorithms are discussed in detail. 

{\bf Chapter \ref{chap: PBE}} reviews the PB model with the Poisson-Boltzmann Equation which will be the problem we will be the center of the discussion in the rest of the dissertation. Different type of Molecular surfaces have been reviewed to explain our choice of SES surface to be generated by the MSMS software. A detail description of the previous ADI method \cite{Geng2013_Fully} has been incorporated. At the end of the chapter the development of the software package REG-GFM-MSMS from ADI-MSMS has been described. 

{\bf Chapter \ref{chap:opt_split}} reviews the two-component regularization and its incorporation with the pseudo-transient approaches. Three types of operator splitting methods have been proposed in this chapter to the solve the PBE. An analytical background of the calculation of the solvation energy from the solution of the PBE has been discussed.

{\bf Chapter \ref{chap:new_GFM}} introduces a modified version of the Ghost Fluid Method (GFM) and its detail derivation. It also covers a background on other GFM methods.

{\bf Chapter \ref{chap:num_vald}} examines the numerical validation for the proposed three methods in chapter \ref{chap:opt_split} for the krikwood sphere problem and other biological problems. For the benchmark problem  several tests have been performed to test the stability, the spatial convergence and the temporal convergence. Similar tests have performed to calculate the solvation energy for a collection 24 proteins to calculate their solvation energies. To calculate other type of biological feature of proteins, the binding energy of HIV viral replication has been performed. As the last test the salt effects on the binding energy of several proteins has been calculated to compare with the experimental datas available.    

{\bf Chapter \ref{chap: conclusions}} finally summarizes the findings made in the this dissertation and proposes some opportunities for the future work.  